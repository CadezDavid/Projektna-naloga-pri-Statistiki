\documentclass{article}
\usepackage{amsmath, amsthm, amsfonts, amssymb}
\usepackage{tikz} 
\usepackage{cite} 
\usepackage{enumitem}
\usepackage[margin=10pt,font=small,labelfont=bf]{caption}

\DeclareMathOperator{\mse}{MSE}
\DeclareMathOperator{\var}{var}
\DeclareMathOperator{\Var}{Var}
\DeclareMathOperator{\se}{SE}

\title{Projektna naloga iz Statistike}
\author{David Čadež}
\date{}

\begin{document}

\maketitle

\section{Prva naloga}

\begin{enumerate}[label=\alph*)]
    \item Na vzorcu $200$ družin smo ocenili delež družin v populaciji, pri
        katerih vodja gospodinjstva nima srednješolske izobrazbe. Cenilka za to
        je bilo enostavno povprečje.
    \item Standardno napako ocenimo z nepristransko cenilko $\hat{SE}_+^2$
%    \item Ker je cenilka za delež iz a) dela naloge nepristranska, je srednja
%        kvadratična napaka enaka kar varianci cenilke $\hat{d}$.
%        Izračunamo torej
%        \[
%            \mse(\hat{d}) = \var(\hat{d}) = \frac{\sigma^2}{n} \left( \frac{N - n}{N - 1} \right)
%        \]
%        Varianca $\sigma$ je enaka $p (1 - p)$, ker je $d$ indikator dogodka. To
%        vstavimo v enačbo in dobimo oceno za $SE$
%        \[
%            \se(\hat{d}) = \sqrt{\var(\hat{d})} = \sqrt{\frac{\hat{p} (1 -
%            \hat{p})}{n} \frac{N - n}{N - 1}}.
%        \]
%
    \item Pravi delež ljudi, ki nimajo srednješolske izobrazbe je enak $
\end{enumerate}

\section{Druga naloga}
\section{Tretja naloga}

\nocite{*}
\bibliographystyle{siam}
\bibliography{viri}{}

\end{document}
